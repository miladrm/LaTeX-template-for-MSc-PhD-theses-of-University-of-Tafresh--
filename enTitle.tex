% !TeX root=main.tex
% در این فایل، عنوان پایان‌نامه، مشخصات خود و چکیده پایان‌نامه را به انگلیسی، وارد کنید.

%%%%%%%%%%%%%%%%%%%%%%%%%%%%%%%%%%%%
\baselineskip=.6cm
\begin{latin}
\latinuniversity{Tafresh University}
\latinfaculty{Ministry of Science, Research and Technology}
\latindepartment{Industrial Engineering Department}
\latinsath{MSc Thesis}
\latintitle{A Risk Adjusted Supplier Selection Model}
\firstlatinsupervisor{Dr. Mohammad Saffari}
%\secondlatinsupervisor{Second Supervisor}
\firstlatinadvisor{Dr. Ali Hossein Mirzaie}
%\secondlatinadvisor{Second Advisor}
\latinname{Miald}
\latinsurname{Rafiee Mashhadi Farahani}
\latinthesisdate{2018}
\latinkeywords{ Supplier Selection; Risk Quantification; Value at Risk; Miss the Target; Multi Objective Optimization; Goal Programming; Nondominated Sorting Genetic Algorithm 2 (NSGA-2) }
\en-abstract{
In the global economy, firms are frequently searching their supplier base around the world to find opportunities for reduce supply chain costs. However, singular emphasis on supply chain cost can make the supply chain brittle and susceptible to the risk of disruptions. In this thesis, we used two different types of risk models, Value at Risk (VaR) and Miss the Target (MtT). VaR type risks are used to model less frequent event which disrupt operations at suppliers and can bring severe impacts to buyers (e.g. labour strike, terrorist attack, natural disaster, etc.). Miss the Target (MtT) type risks, on the other hand, are used to model events that might happen more frequently at suppliers with lesser damage to buyers (e.g. late delivery, missing quality requirement, etc.). In this thesis we propose two multiobjective mathematical models under two different purchasing strategies. The first strategy, called "single sourcing", assumes that the buyer assigns an order for a product to one and only one supplier; that is, order splitting among suppliers is not allowed. The second strategy, called "multiple sourcing" is a generalization of the single sourcing model where the buyer can split  an order among a predetermined number of suppliers. In first model, we have presented a model for assigning a group of primary suppliers and several backup suppliers to a buyer. primary suppliers are those suppliers that ship orders to buyers, whereas backup suppliers are used only when a primary supplier faces disruption. In second model, we used product bundling as a form of quantity discount. Bundling is a form of discount where the final price of a product depends on the quantities of different products ordered. Both models consider three objectives: minimizing total cost, Value at risk value, and Miss the Target value of purchased items. The multiobjective models are solved using "Goal Programming" and "Nondominated Sorting Genetic Algorithm 2 (NSGA-2)" and their solution are illustrated.
}

\latinfirstPage

\end{latin}
\clearpage~
\thispagestyle{empty}

