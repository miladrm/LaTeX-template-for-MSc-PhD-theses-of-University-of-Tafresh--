% !TeX root=main.tex
% در این فایل، عنوان پایان‌نامه، مشخصات خود، متن تقدیمی‌، ستایش، سپاس‌گزاری و چکیده پایان‌نامه را به فارسی، وارد کنید.
% توجه داشته باشید که جدول حاوی مشخصات پروژه/پایان‌نامه/رساله و همچنین، مشخصات داخل آن، به طور خودکار، درج می‌شود.
%%%%%%%%%%%%%%%%%%%%%%%%%%%%%%%%%%%%
% دانشگاه خود را وارد کنید
\university{دانشگاه تفرش}
% دانشکده، آموزشکده و یا پژوهشکده  خود را وارد کنید
\faculty{وزارت علوم، تحقیقات و فناوری }
% گروه آموزشی خود را وارد کنید

\sath{پایان نامه کارشناسی ارشد}

\department{دانشکده مهندسی صنایع}
% گروه آموزشی خود را وارد کنید
\subject{مهندسی صنایع}
% گرایش خود را وارد کنید
\field{زنیجیره‌تامین}
% عنوان پایان‌نامه را وارد کنید
\title{مدل انتخاب تامین‌کننده با در نظر‌گرفتن ریسک}
% نام استاد(ان) راهنما را وارد کنید
\firstsupervisor{آقای دکتر محمد صفاری}
%\secondsupervisor{}
% نام استاد(دان) مشاور را وارد کنید. چنانچه استاد مشاور ندارید، دستور پایین را غیرفعال کنید.
\firstadvisor{آقای دکتر علی حسین میرزایی }
%\secondadvisor{استاد مشاور دوم}
% نام دانشجو را وارد کنید
\name{میلاد}
% نام خانوادگی دانشجو را وارد کنید
\surname{رفیعی مشهدی فراهانی}
% شماره دانشجویی دانشجو را وارد کنید
\studentID{954151}
% تاریخ پایان‌نامه را وارد کنید
\thesisdate{۱۳۹7}
% به صورت پیش‌فرض برای پایان‌نامه‌های کارشناسی تا دکترا به ترتیب از عبارات «پروژه»، «پایان‌نامه» و »رساله» استفاده می‌شود؛ اگر  نمی‌پسندید هر عنوانی را که مایلید در دستور زیر قرار داده و آنرا از حالت توضیح خارج کنید.
%\projectLabel{پایان‌نامه}

% به صورت پیش‌فرض برای عناوین مقاطع تحصیلی کارشناسی تا دکترا به ترتیب از عبارات «کارشناسی»، «کارشناسی ارشد» و »دکترا» استفاده می‌شود؛ اگر  نمی‌پسندید هر عنوانی را که مایلید در دستور زیر قرار داده و آنرا از حالت توضیح خارج کنید.
%\degree{}

%\firstPage
\besmPage
\firstPage
%\davaranPage

%\vspace{.5cm}
% در این قسمت اسامی اساتید راهنما، مشاور و داور باید به صورت دستی وارد شوند
%\renewcommand{\arraystretch}{1.2}


\esalatPage
%\mojavezPage


% چنانچه مایل به چاپ صفحات «تقدیم»، «نیایش» و «سپاس‌گزاری» در خروجی نیستید، خط‌های زیر را با گذاشتن ٪  در ابتدای آنها غیرفعال کنید.
 % پایان‌نامه خود را تقدیم کنید!

 \newpage
\thispagestyle{empty}
{\Large تقدیم به:}\\
\begin{flushleft}
{\huge
عزیزانم\\
\vspace{7mm}


}
\end{flushleft}


% سپاس‌گزاری
{\nastaliq\begin{acknowledgementpage}
سپاس خداوندگار حکیم را که با لطف بی‌کران خود، آدمی را زیور عقل آراست.


در آغاز وظیفه‌  خود  می‌دانم از زحمات بی‌دریغ استاد  راهنمای خود،  جناب آقای دکتر محمد صفاری، صمیمانه تشکر و  قدردانی کنم  که قطعاً بدون راهنمایی‌های ارزنده‌  ایشان، این مجموعه  به انجام  نمی‌رسید.

از جناب  آقای  دکتر میرزایی   که زحمت  مطالعه و مشاوره‌  این رساله را تقبل  فرمودند و در آماده سازی  این رساله، به نحو احسن اینجانب را مورد راهنمایی قرار دادند، کمال امتنان را دارم.

بر خود فرض می‌دانم از اساتید ارجمند دکتر اشجری و دکتر گلمکانی تشکر و قدردانی کنم؛ اینجانب در محضر این اساتید تلمذ کرده و از کلاس درس ایشان بهره‌مند شده‌ام.

همچنین از کارشناس محترم گروه مهندسی صنایع دانشگاه تفرش؛ سرکار خانم دواتگری، که همیشه از روی صمیمیت و مهربانی با دانشجویان برخورد می‌کنند، نهایت تشکر و قدردانی را می‌نمایم.

 در پایان، بوسه می‌زنم بر دستان خداوندگاران مهر و مهربانی، پدر و مادر عزیزم و بعد از خدا، ستایش می‌کنم وجود مقدس‌شان را و تشکر می‌کنم از خانواده عزیزم به پاس عاطفه سرشار و گرمای امیدبخش وجودشان، که بهترین پشتیبان من بودند.
% با استفاده از دستور زیر، امضای شما، به طور خودکار، درج می‌شود.
\signature 
\end{acknowledgementpage}}
%%%%%%%%%%%%%%%%%%%%%%%%%%%%%%%%%%%%
% کلمات کلیدی پایان‌نامه را وارد کنید
\keywords{
انتخاب تامین‌کننده، کمی‌سازی ریسک، ارزش در معرض ریسک، ارزش از دست رفته، بهینه‌سازی چندهدفه، برنامه‌ریزی آرمانی، الگوریتم ژنتیک با مرتب‌سازی نامغلوب (نسخه دوم)
}
%چکیده پایان‌نامه را وارد کنید، برای ایجاد پاراگراف جدید از \\ استفاده کنید. اگر خط خالی دشته باشید، خطا خواهید گرفت.
\fa-abstract{در اقتصاد جهانی، شرکت‌ها مرتباً به دنبال تامین‌کنندگان خود در سرتاسر جهان به منظور دستیابی به فرصتی برای کاهش هزینه‌های زنجیره‌تامین هستند. با این حال، تاکید مفرد بر روی هزینه‌های زنجیره‌تامین، می‌تواند زنجیره‌تامین را نسبت به ریسک اختلالات، شکننده و آسیب‌پذیر کند. در این پایان‌نامه، ما از دو نوع مختلف از ریسک استفاده می‌کنیم؛ ارزش در معرض ریسک (\lr{VaR})  و ارزش از دست رفته (\lr{MtT}).
ریسک‌های نوع 
\lr{VaR}
برای مدل کردن حوادث با رخداد کم است که موجب اختلال در عملیات‌های تامین‌کننده و وارد آمدن زیان شدید به خریداران می‌شود (مانند: اعتصابات کارگری، حملات تروریستی، حوادث طبیعی و ...)
. از طرف دیگر، ریسک‌های از نوع ارزش از دست رفته 
(\lr{MtT})
برای مدل کردن حوادث با رخداد بالا برای تامین‌کننده و زیان کمتر به خریدار (مانند: دیرکرد تحویل، از دست دادن الزامات کیفیت و ...)
مورد استفاده قرار می‌گیرند. 
در این پایان‌نامه، ما دو مدل ریاضی چندهدفه تحت دو استراتژی خرید مختلف، ارائه کرده‌ایم. در اولین استراتژی که تک منبعی نامیده می‌شود، فرض می‌شود که خریدار مقدار سفارش برای یک محصول را به یک و فقط یک تامین‌کننده، تخصیص می‌دهد؛ به این معنی که تقسیم کردن سفارش بین تامین‌کنندگان مجاز نمی‌باشد. دومین استراتژی، استراتژی چندمنبعی است که تعمیمی است از مدل تک‌منبعی که خریدار می‌تواند یک سفارش را میان تعدادی از تامین‌کنندگان تقسیم کند. در اولین مدل، ما یک مدل برای تخصیص یک گروهی از تامین‌کنندگان اصلی و چندین تامین‌کننده پشتیبان به یک خریدار، ارائه کرده‌ایم. تامین‌کنندگان اصلی، آن‌هایی هستند که سفارشات را به خریداران تحویل می‌دهند، در حالی که تامین‌کنندگان پشتیبان، تنها زمانی مورد استفاده قرار می‌گیرند که تامین‌کننده اصلی با اختلال مواجه شده است. در دومین مدل، ما از تخفیف بسته‌بندی به عنوان یک شکلی از تخفیف مقداری استفاده کرده‌ایم. بسته‌بندی یک فرمی از تخفیف است که در آن قیمت نهایی یک محصول، به مقادیر سفارش داده شده از محصولات دیگر بستگی دارد. هر دو مدل ارائه شده، دارای سه تابع هدف هستند: حداقل‌سازی هزینه‌ کل، مقدار ارزش در معرض ریسک و ارزش از دست رفته، موارد خریداری شده. این مدل‌ها با استفاده از روش «برنامه‌ریزی آرمانی» و «الگوریتم ژنتیک با مرتب‌سازی نامغلوب نسخه دوم 
\lr{NSGA-2}
» حل شده‌اند و جواب‌هایشان نشان داده شده است.
}



\abstractPage

\newpage\clearpage