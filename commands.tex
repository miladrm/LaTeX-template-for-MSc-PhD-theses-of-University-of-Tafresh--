% در این فایل، دستورها و تنظیمات مورد نیاز، آورده شده است.
%-------------------------------------------------------------------------------------------------------------------

% در ورژن جدید زی‌پرشین برای تایپ متن‌های ریاضی، این سه بسته، حتماً باید فراخوانی شود
\usepackage{amsthm,amssymb,amsmath}
% بسته‌ای برای تنطیم حاشیه‌های بالا، پایین، چپ و راست صفحه
\usepackage{multirow}
\usepackage{titlesec}

\usepackage{ptext}
\usepackage{titletoc}
\usepackage{ptext}
\usepackage{caption}

\usepackage{enumerate}
\usepackage{listings}


\usepackage[inline,shortlabels]{enumitem}  




\usepackage[top=2cm, bottom=2.5cm, left=2cm, right=3cm]{geometry}
% بسته‌‌ای برای ظاهر شدن شکل‌ها و تصاویر متن
\usepackage{graphicx}
% بسته‌ای برای رسم کادر
\usepackage{framed} 
% بسته‌‌ای برای چاپ شدن خودکار تعداد صفحات در صفحه «معرفی پایان‌نامه»
\usepackage{lastpage}
% بسته‌ و دستوراتی برای ایجاد لینک‌های رنگی با امکان جهش
%\usepackage[pagebackref=false,colorlinks,linkcolor=blue,citecolor=blue]{hyperref}
% چنانچه قصد پرینت گرفتن نوشته خود را دارید، خط بالا را غیرفعال و  از دستور زیر استفاده کنید چون در صورت استفاده از دستور زیر‌‌، 
% لینک‌ها به رنگ سیاه ظاهر خواهند شد که برای پرینت گرفتن، مناسب‌تر است
%\usepackage[pagebackref=false]{hyperref}
% بسته‌ لازم برای تنظیم سربرگ‌ها
\usepackage{fancyhdr}
%
\usepackage{setspace}
\usepackage{algorithm}
\usepackage{algorithmic}
\usepackage{subfigure}
\usepackage[subfigure]{tocloft}
\renewcommand{\labelitemi}{$\bullet$}
\usepackage{bidiftnxtra}
\usepackage{tablefootnote}
\usepackage{multirow}
%\usepackage{caption}

% بسته‌ای برای ظاهر شدن «مراجع» و «نمایه» در فهرست مطالب
\usepackage[nottoc]{tocbibind}
% دستورات مربوط به ایجاد نمایه
\usepackage{makeidx}
\makeindex
%%%%%%%%%%%%%%%%%%%%%%%%%%
% فراخوانی بسته زی‌پرشین و تعریف قلم فارسی و انگلیسی
\usepackage{xepersian}
\setlength{\parskip}{6pt}
\baselineskip=1 cm
\settextfont[Scale=1.3]{IRMitra.ttf}
\setlatintextfont[Scale=1.1]{Times New Roman}

%%%%%%%%%%%%%%%%%%%%%%%%%%
% چنانچه می‌خواهید اعداد در فرمول‌ها، انگلیسی باشد، خط زیر را غیرفعال کنید
\setdigitfont[Scale=1.3]{IRNazanin.ttf}%{Persian Modern}

%%%%%%%%%%%%%%%%%%%%%%%%%%
%%%%%%%%%%%%%%%%%%%%%%%%%%
% تعریف قلم‌های فارسی و انگلیسی اضافی برای استفاده در بعضی از قسمت‌های متن
\defpersianfont\titlefont[Scale=1]{IRTitr.ttf}
%\defpersianfont\titr[Scale=1.4]{XB Titre}
\defpersianfont\titra[Scale=1.4]{IRTitr.ttf}
\defpersianfont\titrb[Scale=2.5]{IRTitr.ttf}
\defpersianfont\titrc[Scale=2.2]{IRTitr.ttf}
\defpersianfont\titrd[Scale=2.6]{IRTitr.ttf}
\defpersianfont\titre[Scale=2.1]{IRTitr.ttf}
\defpersianfont\mitrafonta[Scale=1.9]{IRMitraBold.ttf}
\defpersianfont\mitrafontb[Scale=1.3]{IRMitraBold.ttf}

\defpersianfont\nazfont[Scale=1.3]{IRNazanin.ttf}
\defpersianfont\nazfontbold[Scale=1.3]{IRNazaninBold.ttf}

\defpersianfont\tzarfont[Scale=1.2]{IRZarBold.ttf}
\defpersianfont\zarfont[Scale=1.3]{IRZar.ttf}
\defpersianfont\mitfont[Scale=1.2]{IRMitra.ttf}
% \defpersianfont\iranic[Scale=1.1]{XB Zar Oblique}%Italic}%
\defpersianfont\nastaliq[Scale=1.2]{IranNastaliq}


\defpersianfont\headfont[Scale=1.8]{IRLotusBold.ttf}
\defpersianfont\sectionfont[Scale=1.6]{IRLotusBold.ttf}
\defpersianfont\subsectionfont[Scale=1.4]{IRLotusBold.ttf}
\defpersianfont\subsubsectionfont[Scale=1.3]{IRLotusBold.ttf}

%%%%%%%%%%%%%%%%%%%%%%%%%%%%%
%%%%%%%%%%%%%%%%%%%%%%%%%%%%%
\defpersianfont\cpf[Scale=1.2]{IRLotusBold.ttf}
\DeclareCaptionFont{cpf}{\cpf}
\captionsetup{textfont=cpf, labelfont=cpf}





%%%%%%%%%%%%%%%%%%%%%%%%%%
% دستوری برای حذف کلمه «چکیده»
\renewcommand{\abstractname}{}
% دستوری برای حذف کلمه «abstract»
%\renewcommand{\latinabstract}{}
% دستوری برای تغییر نام کلمه «اثبات» به «برهان»
\renewcommand\proofname{\textbf{برهان}}
% دستوری برای تغییر نام کلمه «کتاب‌نامه» به «مراجع»
\renewcommand{\bibname}{مراجع}
% دستوری برای تعریف واژه‌نامه انگلیسی به فارسی
\newcommand\persiangloss[2]{#1\dotfill\lr{#2}\\}
% دستوری برای تعریف واژه‌نامه فارسی به انگلیسی 
\newcommand\englishgloss[2]{#2\dotfill\lr{#1}\\}
% تعریف دستور جدید «\پ» برای خلاصه‌نویسی جهت نوشتن عبارت «پروژه/پایان‌نامه/رساله»
\newcommand{\پ}{پروژه/پایان‌نامه/رساله }

%\newcommand\BackSlash{\char`\\}

%%%%%%%%%%%%%%%%%%%%%%%%%%
\SepMark{-}

% تعریف و نحوه ظاهر شدن عنوان قضیه‌ها، تعریف‌ها، مثال‌ها و ...
\theoremstyle{definition}
\newtheorem{definition}{\mitrafontb{تعریف}}[section]
\theoremstyle{theorem}
\newtheorem{theorem}[definition]{قضیه}
\newtheorem{lemma}[definition]{لم}
\newtheorem{proposition}[definition]{گزاره}
\newtheorem{corollary}[definition]{نتیجه}
\newtheorem{remark}[definition]{ملاحظه}
\theoremstyle{definition}
\newtheorem{example}[definition]{مثال}

%\renewcommand{\theequation}{\thechapter-\arabic{equation}}
%\def\bibname{مراجع}
\numberwithin{algorithm}{chapter}
\def\listalgorithmname{فهرست الگوریتم‌ها}
\def\listfigurename{فهرست تصاویر}
\def\listtablename{فهرست جداول}

%%%%%%%%%%%%%%%%%%%%%%%%%%%%%%%%%%%%%%%%%%%%%%%
% دستورهایی برای سفارشی کردن سربرگ صفحات
%\csname@twosidetrue\endcsname
\pagestyle{fancy}
\fancyhf{} 
\fancyhead[OL,EL]{\thepage}
\fancyhead[OR,ER]{\small\leftmark}
\renewcommand{\chaptermark}[1]{%
\markboth{#1}{}}
\newcommand{\فا}[1]{#1\index{#1}} 
\newcommand{\ف}{\index} 
\newcommand{\فف}[1]{\index{\lr{#1}}} 
\newcommand{\LineStretch}[1]{\renewcommand{\baselinestretch}{#1}	
\settextfont[Scale=1.1]{IRLotusBold.ttf}
} 
%%%%%%%%%%%%%%%%%%%%%%%%%%%%%%%%%%%%%%%%%%%%%%%%









%%%%%%%%%%%%%%%%%%%%%%%%%%%%
% دستورهایی برای سفارشی کردن سربرگ صفحات
% \newcommand{\SetHeader}{
% \csname@twosidetrue\endcsname
% \pagestyle{fancy}
% \fancyhf{} 
% \fancyhead[OL,EL]{\thepage}
% \fancyhead[OR]{\small\rightmark}
% \fancyhead[ER]{\small\leftmark}
% \renewcommand{\chaptermark}[1]{%
% \markboth{\thechapter-\ #1}{}}
% }
%%%%%%%%%%%%5
%\def\MATtextbaseline{1.5}
%\renewcommand{\baselinestretch}{\MATtextbaseline}
\doublespacing
%%%%%%%%%%%%%%%%%%%%%%%%%%%%%
% دستوراتی برای شخصی سازی فهرست مطالب با توجه به شیوه نامه دانشگاه تفرش

\newlength\mylenprt
\newlength\mylenchp
\newlength\mylenapp

\renewcommand\cftpartpresnum{\partname~}
\renewcommand\cftchappresnum{\chaptername~}
\renewcommand\cftchapaftersnum{:}

% شخصی سازی فصل ها در فهرست مطالب
%\renewcommand{\cftchapfont}{\fontsize{15pt}{15pt}\nazfontbold}
%شخصی سازی بخشها در فهرست مطالب
\renewcommand{\cftsecfont}{\nazfont}
%شخصی سازی کلمه «فهرست مطالب» 

\renewcommand{\cfttoctitlefont}{\hfill\nazfontbold} 
\renewcommand{\cftaftertoctitle}{\hfill}
%شخصی سازی کلمه فهرست تصاویر
\renewcommand{\cftloftitlefont}{\hfill\nazfontbold} 
\renewcommand{\cftafterloftitle}{\hfill}
%شخصی سازی کلمه فهرست جداول
\renewcommand{\cftlottitlefont}{\hfill\nazfontbold} 
\renewcommand{\cftafterlottitle}{\hfill}





\settowidth\mylenprt{\cftpartfont\cftpartpresnum\cftpartaftersnum}
\settowidth\mylenchp{\cftchapfont\cftchappresnum\cftchapaftersnum}
\settowidth\mylenapp{\cftchapfont\appendixname~\cftchapaftersnum}
\addtolength\mylenprt{\cftpartnumwidth}
\addtolength\mylenchp{\cftchapnumwidth}
\addtolength\mylenapp{\cftchapnumwidth}

\setlength\cftpartnumwidth{\mylenprt}
\setlength\cftchapnumwidth{\mylenchp}	

\makeatletter
\bidi@patchcmd{\@makechapterhead}{\thechapter}{\tartibi{chapter}}{}{}
\bidi@patchcmd{\chaptermark}{\thechapter}{\tartibi{chapter}}{}{}
\makeatother


\titlecontents{chapter}% <section-type>
[0pt]% <left>
{}% <above-code>
{\nazfontbold\chaptername\ \tartibinumeral{\thecontentslabel}:\quad\hfil}% <numbered-entry-format>
{}% <numberless-entry-format>
{\bfseries\hfill\contentspage}% <filler-page-format>

%%%%%%%%%%%%%%%%%%%%%%%%%%%%%%%%%%%%%%%%%%%%%%%%%%%%%%%%
%%%%%%%%%%%%%%%%%%%%%%%%%%%%%%%%%%%%%%%%%%%%%%%%%%%%%%%%
%%%%%%%%%%



\titleformat
{\chapter} %comand
[display] %Shape
{\headfont}    %format
{\vspace{-2.2cm}\centering \chaptername \hspace*{.1mm} \tartibi{chapter}} %label
{0ex} %Sep
{\vspace{.6cm} \centering} % before code
[\vspace{1.5cm}] % after code






\titleformat{\section}
  {\sectionfont}          
  {\thesection}{1em}{} 

\titleformat{\subsection}
  {\subsectionfont}          
  {\thesubsection}{1em}{} 

\titleformat{\subsubsection}
  {\subsubsectionfont}          
  {\thesubsubsection}{1em}{} 




