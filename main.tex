% !TEX TS-program = XeLaTeX

% Commands for running this example:
% 	 xelatex main
% 	 bibtex8 -W -c cp1256fa main
%      xindy -L persian -C utf8 -M texindy main
% 	 xelatex main
% 	 xelatex main
% End of Commands

%        نمونه پایان‌نامه آماده شده با استفاده از کلاس Tafresh-Thesis
%		 میلاد رفیعی (miladrafiee2013@gmail.com)، کارشناسی ارشد مهندسی صنایع، دانشگاه تفرش
% 		گروه پارسی‌لاتک  http://www.parsilatex.com
%        این نسخه، بر اساس نسخه‌ 0.4 از کلاس Tabriz_Thesis آقای وحید دامن‌افشان آماده شده است. http://damanafshan.tk
%        
%        
%%%%%%%%%%%%%%%%%%%%%%%%%%%%%%%%%%%%%%%%%%%%%%%%%%%%%%%%%%%%%
%        اگر قصد نوشتن پروژه کارشناسی را دارید، در خط زیر به جای msc، کلمه bsc و اگر قصد نوشتن پروژه دکترا
%        را دارید، کلمه phd را قرار دهید. کلیه تنظیمات لازم، به طور خودکار، اعمال می‌شود.

%        اگر مایلید پایان‌نامه شما دورو باشد به جای oneside  در خط زیر از twoside استفاده کنید
\documentclass[oneside,openany,msc]{Tafresh-Thesis}

% مشخصات پایان‌نامه را در فایلهای faTitle و enTitle وارد نمایید.

%       فایل commands.tex را مطالعه کنید؛ چون دستورات مربوط به فراخوانی بسته زی‌پرشین 
%       و دیگر بسته‌ها و ... در این فایل قرار دارد و بهتر است که با نحوه استفاده از آنها آشنا شوید.
% در این فایل، دستورها و تنظیمات مورد نیاز، آورده شده است.
%-------------------------------------------------------------------------------------------------------------------

% در ورژن جدید زی‌پرشین برای تایپ متن‌های ریاضی، این سه بسته، حتماً باید فراخوانی شود
\usepackage{amsthm,amssymb,amsmath}
% بسته‌ای برای تنطیم حاشیه‌های بالا، پایین، چپ و راست صفحه
\usepackage{multirow}
\usepackage{titlesec}

\usepackage{ptext}
\usepackage{titletoc}
\usepackage{ptext}
\usepackage{caption}

\usepackage{enumerate}
\usepackage{listings}


\usepackage[inline,shortlabels]{enumitem}  




\usepackage[top=2cm, bottom=2.5cm, left=2cm, right=3cm]{geometry}
% بسته‌‌ای برای ظاهر شدن شکل‌ها و تصاویر متن
\usepackage{graphicx}
% بسته‌ای برای رسم کادر
\usepackage{framed} 
% بسته‌‌ای برای چاپ شدن خودکار تعداد صفحات در صفحه «معرفی پایان‌نامه»
\usepackage{lastpage}
% بسته‌ و دستوراتی برای ایجاد لینک‌های رنگی با امکان جهش
%\usepackage[pagebackref=false,colorlinks,linkcolor=blue,citecolor=blue]{hyperref}
% چنانچه قصد پرینت گرفتن نوشته خود را دارید، خط بالا را غیرفعال و  از دستور زیر استفاده کنید چون در صورت استفاده از دستور زیر‌‌، 
% لینک‌ها به رنگ سیاه ظاهر خواهند شد که برای پرینت گرفتن، مناسب‌تر است
%\usepackage[pagebackref=false]{hyperref}
% بسته‌ لازم برای تنظیم سربرگ‌ها
\usepackage{fancyhdr}
%
\usepackage{setspace}
\usepackage{algorithm}
\usepackage{algorithmic}
\usepackage{subfigure}
\usepackage[subfigure]{tocloft}
\renewcommand{\labelitemi}{$\bullet$}
\usepackage{bidiftnxtra}
\usepackage{tablefootnote}
\usepackage{multirow}
%\usepackage{caption}

% بسته‌ای برای ظاهر شدن «مراجع» و «نمایه» در فهرست مطالب
\usepackage[nottoc]{tocbibind}
% دستورات مربوط به ایجاد نمایه
\usepackage{makeidx}
\makeindex
%%%%%%%%%%%%%%%%%%%%%%%%%%
% فراخوانی بسته زی‌پرشین و تعریف قلم فارسی و انگلیسی
\usepackage{xepersian}
\setlength{\parskip}{6pt}
\baselineskip=1 cm
\settextfont[Scale=1.3]{IRMitra.ttf}
\setlatintextfont[Scale=1.1]{Times New Roman}

%%%%%%%%%%%%%%%%%%%%%%%%%%
% چنانچه می‌خواهید اعداد در فرمول‌ها، انگلیسی باشد، خط زیر را غیرفعال کنید
\setdigitfont[Scale=1.3]{IRNazanin.ttf}%{Persian Modern}

%%%%%%%%%%%%%%%%%%%%%%%%%%
%%%%%%%%%%%%%%%%%%%%%%%%%%
% تعریف قلم‌های فارسی و انگلیسی اضافی برای استفاده در بعضی از قسمت‌های متن
\defpersianfont\titlefont[Scale=1]{IRTitr.ttf}
%\defpersianfont\titr[Scale=1.4]{XB Titre}
\defpersianfont\titra[Scale=1.4]{IRTitr.ttf}
\defpersianfont\titrb[Scale=2.5]{IRTitr.ttf}
\defpersianfont\titrc[Scale=2.2]{IRTitr.ttf}
\defpersianfont\titrd[Scale=2.6]{IRTitr.ttf}
\defpersianfont\titre[Scale=2.1]{IRTitr.ttf}
\defpersianfont\mitrafonta[Scale=1.9]{IRMitraBold.ttf}
\defpersianfont\mitrafontb[Scale=1.3]{IRMitraBold.ttf}

\defpersianfont\nazfont[Scale=1.3]{IRNazanin.ttf}
\defpersianfont\nazfontbold[Scale=1.3]{IRNazaninBold.ttf}

\defpersianfont\tzarfont[Scale=1.2]{IRZarBold.ttf}
\defpersianfont\zarfont[Scale=1.3]{IRZar.ttf}
\defpersianfont\mitfont[Scale=1.2]{IRMitra.ttf}
% \defpersianfont\iranic[Scale=1.1]{XB Zar Oblique}%Italic}%
\defpersianfont\nastaliq[Scale=1.2]{IranNastaliq}


\defpersianfont\headfont[Scale=1.8]{IRLotusBold.ttf}
\defpersianfont\sectionfont[Scale=1.6]{IRLotusBold.ttf}
\defpersianfont\subsectionfont[Scale=1.4]{IRLotusBold.ttf}
\defpersianfont\subsubsectionfont[Scale=1.3]{IRLotusBold.ttf}

%%%%%%%%%%%%%%%%%%%%%%%%%%%%%
%%%%%%%%%%%%%%%%%%%%%%%%%%%%%
\defpersianfont\cpf[Scale=1.2]{IRLotusBold.ttf}
\DeclareCaptionFont{cpf}{\cpf}
\captionsetup{textfont=cpf, labelfont=cpf}





%%%%%%%%%%%%%%%%%%%%%%%%%%
% دستوری برای حذف کلمه «چکیده»
\renewcommand{\abstractname}{}
% دستوری برای حذف کلمه «abstract»
%\renewcommand{\latinabstract}{}
% دستوری برای تغییر نام کلمه «اثبات» به «برهان»
\renewcommand\proofname{\textbf{برهان}}
% دستوری برای تغییر نام کلمه «کتاب‌نامه» به «مراجع»
\renewcommand{\bibname}{مراجع}
% دستوری برای تعریف واژه‌نامه انگلیسی به فارسی
\newcommand\persiangloss[2]{#1\dotfill\lr{#2}\\}
% دستوری برای تعریف واژه‌نامه فارسی به انگلیسی 
\newcommand\englishgloss[2]{#2\dotfill\lr{#1}\\}
% تعریف دستور جدید «\پ» برای خلاصه‌نویسی جهت نوشتن عبارت «پروژه/پایان‌نامه/رساله»
\newcommand{\پ}{پروژه/پایان‌نامه/رساله }

%\newcommand\BackSlash{\char`\\}

%%%%%%%%%%%%%%%%%%%%%%%%%%
\SepMark{-}

% تعریف و نحوه ظاهر شدن عنوان قضیه‌ها، تعریف‌ها، مثال‌ها و ...
\theoremstyle{definition}
\newtheorem{definition}{\mitrafontb{تعریف}}[section]
\theoremstyle{theorem}
\newtheorem{theorem}[definition]{قضیه}
\newtheorem{lemma}[definition]{لم}
\newtheorem{proposition}[definition]{گزاره}
\newtheorem{corollary}[definition]{نتیجه}
\newtheorem{remark}[definition]{ملاحظه}
\theoremstyle{definition}
\newtheorem{example}[definition]{مثال}

%\renewcommand{\theequation}{\thechapter-\arabic{equation}}
%\def\bibname{مراجع}
\numberwithin{algorithm}{chapter}
\def\listalgorithmname{فهرست الگوریتم‌ها}
\def\listfigurename{فهرست تصاویر}
\def\listtablename{فهرست جداول}

%%%%%%%%%%%%%%%%%%%%%%%%%%%%%%%%%%%%%%%%%%%%%%%
% دستورهایی برای سفارشی کردن سربرگ صفحات
%\csname@twosidetrue\endcsname
\pagestyle{fancy}
\fancyhf{} 
\fancyhead[OL,EL]{\thepage}
\fancyhead[OR,ER]{\small\leftmark}
\renewcommand{\chaptermark}[1]{%
\markboth{#1}{}}
\newcommand{\فا}[1]{#1\index{#1}} 
\newcommand{\ف}{\index} 
\newcommand{\فف}[1]{\index{\lr{#1}}} 
\newcommand{\LineStretch}[1]{\renewcommand{\baselinestretch}{#1}	
\settextfont[Scale=1.1]{IRLotusBold.ttf}
} 
%%%%%%%%%%%%%%%%%%%%%%%%%%%%%%%%%%%%%%%%%%%%%%%%









%%%%%%%%%%%%%%%%%%%%%%%%%%%%
% دستورهایی برای سفارشی کردن سربرگ صفحات
% \newcommand{\SetHeader}{
% \csname@twosidetrue\endcsname
% \pagestyle{fancy}
% \fancyhf{} 
% \fancyhead[OL,EL]{\thepage}
% \fancyhead[OR]{\small\rightmark}
% \fancyhead[ER]{\small\leftmark}
% \renewcommand{\chaptermark}[1]{%
% \markboth{\thechapter-\ #1}{}}
% }
%%%%%%%%%%%%5
%\def\MATtextbaseline{1.5}
%\renewcommand{\baselinestretch}{\MATtextbaseline}
\doublespacing
%%%%%%%%%%%%%%%%%%%%%%%%%%%%%
% دستوراتی برای شخصی سازی فهرست مطالب با توجه به شیوه نامه دانشگاه تفرش

\newlength\mylenprt
\newlength\mylenchp
\newlength\mylenapp

\renewcommand\cftpartpresnum{\partname~}
\renewcommand\cftchappresnum{\chaptername~}
\renewcommand\cftchapaftersnum{:}

% شخصی سازی فصل ها در فهرست مطالب
%\renewcommand{\cftchapfont}{\fontsize{15pt}{15pt}\nazfontbold}
%شخصی سازی بخشها در فهرست مطالب
\renewcommand{\cftsecfont}{\nazfont}
%شخصی سازی کلمه «فهرست مطالب» 

\renewcommand{\cfttoctitlefont}{\hfill\nazfontbold} 
\renewcommand{\cftaftertoctitle}{\hfill}
%شخصی سازی کلمه فهرست تصاویر
\renewcommand{\cftloftitlefont}{\hfill\nazfontbold} 
\renewcommand{\cftafterloftitle}{\hfill}
%شخصی سازی کلمه فهرست جداول
\renewcommand{\cftlottitlefont}{\hfill\nazfontbold} 
\renewcommand{\cftafterlottitle}{\hfill}





\settowidth\mylenprt{\cftpartfont\cftpartpresnum\cftpartaftersnum}
\settowidth\mylenchp{\cftchapfont\cftchappresnum\cftchapaftersnum}
\settowidth\mylenapp{\cftchapfont\appendixname~\cftchapaftersnum}
\addtolength\mylenprt{\cftpartnumwidth}
\addtolength\mylenchp{\cftchapnumwidth}
\addtolength\mylenapp{\cftchapnumwidth}

\setlength\cftpartnumwidth{\mylenprt}
\setlength\cftchapnumwidth{\mylenchp}	

\makeatletter
\bidi@patchcmd{\@makechapterhead}{\thechapter}{\tartibi{chapter}}{}{}
\bidi@patchcmd{\chaptermark}{\thechapter}{\tartibi{chapter}}{}{}
\makeatother


\titlecontents{chapter}% <section-type>
[0pt]% <left>
{}% <above-code>
{\nazfontbold\chaptername\ \tartibinumeral{\thecontentslabel}:\quad\hfil}% <numbered-entry-format>
{}% <numberless-entry-format>
{\bfseries\hfill\contentspage}% <filler-page-format>

%%%%%%%%%%%%%%%%%%%%%%%%%%%%%%%%%%%%%%%%%%%%%%%%%%%%%%%%
%%%%%%%%%%%%%%%%%%%%%%%%%%%%%%%%%%%%%%%%%%%%%%%%%%%%%%%%
%%%%%%%%%%



\titleformat
{\chapter} %comand
[display] %Shape
{\headfont}    %format
{\vspace{-2.2cm}\centering \chaptername \hspace*{.1mm} \tartibi{chapter}} %label
{0ex} %Sep
{\vspace{.6cm} \centering} % before code
[\vspace{1.5cm}] % after code






\titleformat{\section}
  {\sectionfont}          
  {\thesection}{1em}{} 

\titleformat{\subsection}
  {\subsectionfont}          
  {\thesubsection}{1em}{} 

\titleformat{\subsubsection}
  {\subsubsectionfont}          
  {\thesubsubsection}{1em}{} 






\begin{document}
\clearpage~
\thispagestyle{empty}
\pagenumbering{harfi}
% !TeX root=main.tex
% در این فایل، عنوان پایان‌نامه، مشخصات خود، متن تقدیمی‌، ستایش، سپاس‌گزاری و چکیده پایان‌نامه را به فارسی، وارد کنید.
% توجه داشته باشید که جدول حاوی مشخصات پروژه/پایان‌نامه/رساله و همچنین، مشخصات داخل آن، به طور خودکار، درج می‌شود.
%%%%%%%%%%%%%%%%%%%%%%%%%%%%%%%%%%%%
% دانشگاه خود را وارد کنید
\university{دانشگاه تفرش}
% دانشکده، آموزشکده و یا پژوهشکده  خود را وارد کنید
\faculty{وزارت علوم، تحقیقات و فناوری }
% گروه آموزشی خود را وارد کنید

\sath{پایان نامه کارشناسی ارشد}

\department{دانشکده مهندسی صنایع}
% گروه آموزشی خود را وارد کنید
\subject{مهندسی صنایع}
% گرایش خود را وارد کنید
\field{زنیجیره‌تامین}
% عنوان پایان‌نامه را وارد کنید
\title{مدل انتخاب تامین‌کننده با در نظر‌گرفتن ریسک}
% نام استاد(ان) راهنما را وارد کنید
\firstsupervisor{آقای دکتر محمد صفاری}
%\secondsupervisor{}
% نام استاد(دان) مشاور را وارد کنید. چنانچه استاد مشاور ندارید، دستور پایین را غیرفعال کنید.
\firstadvisor{آقای دکتر علی حسین میرزایی }
%\secondadvisor{استاد مشاور دوم}
% نام دانشجو را وارد کنید
\name{میلاد}
% نام خانوادگی دانشجو را وارد کنید
\surname{رفیعی مشهدی فراهانی}
% شماره دانشجویی دانشجو را وارد کنید
\studentID{954151}
% تاریخ پایان‌نامه را وارد کنید
\thesisdate{۱۳۹7}
% به صورت پیش‌فرض برای پایان‌نامه‌های کارشناسی تا دکترا به ترتیب از عبارات «پروژه»، «پایان‌نامه» و »رساله» استفاده می‌شود؛ اگر  نمی‌پسندید هر عنوانی را که مایلید در دستور زیر قرار داده و آنرا از حالت توضیح خارج کنید.
%\projectLabel{پایان‌نامه}

% به صورت پیش‌فرض برای عناوین مقاطع تحصیلی کارشناسی تا دکترا به ترتیب از عبارات «کارشناسی»، «کارشناسی ارشد» و »دکترا» استفاده می‌شود؛ اگر  نمی‌پسندید هر عنوانی را که مایلید در دستور زیر قرار داده و آنرا از حالت توضیح خارج کنید.
%\degree{}

%\firstPage
\besmPage
\firstPage
%\davaranPage

%\vspace{.5cm}
% در این قسمت اسامی اساتید راهنما، مشاور و داور باید به صورت دستی وارد شوند
%\renewcommand{\arraystretch}{1.2}


\esalatPage
%\mojavezPage


% چنانچه مایل به چاپ صفحات «تقدیم»، «نیایش» و «سپاس‌گزاری» در خروجی نیستید، خط‌های زیر را با گذاشتن ٪  در ابتدای آنها غیرفعال کنید.
 % پایان‌نامه خود را تقدیم کنید!

 \newpage
\thispagestyle{empty}
{\Large تقدیم به:}\\
\begin{flushleft}
{\huge
عزیزانم\\
\vspace{7mm}


}
\end{flushleft}


% سپاس‌گزاری
{\nastaliq\begin{acknowledgementpage}
سپاس خداوندگار حکیم را که با لطف بی‌کران خود، آدمی را زیور عقل آراست.


در آغاز وظیفه‌  خود  می‌دانم از زحمات بی‌دریغ استاد  راهنمای خود،  جناب آقای دکتر محمد صفاری، صمیمانه تشکر و  قدردانی کنم  که قطعاً بدون راهنمایی‌های ارزنده‌  ایشان، این مجموعه  به انجام  نمی‌رسید.

از جناب  آقای  دکتر میرزایی   که زحمت  مطالعه و مشاوره‌  این رساله را تقبل  فرمودند و در آماده سازی  این رساله، به نحو احسن اینجانب را مورد راهنمایی قرار دادند، کمال امتنان را دارم.

بر خود فرض می‌دانم از اساتید ارجمند دکتر اشجری و دکتر گلمکانی تشکر و قدردانی کنم؛ اینجانب در محضر این اساتید تلمذ کرده و از کلاس درس ایشان بهره‌مند شده‌ام.

همچنین از کارشناس محترم گروه مهندسی صنایع دانشگاه تفرش؛ سرکار خانم دواتگری، که همیشه از روی صمیمیت و مهربانی با دانشجویان برخورد می‌کنند، نهایت تشکر و قدردانی را می‌نمایم.

 در پایان، بوسه می‌زنم بر دستان خداوندگاران مهر و مهربانی، پدر و مادر عزیزم و بعد از خدا، ستایش می‌کنم وجود مقدس‌شان را و تشکر می‌کنم از خانواده عزیزم به پاس عاطفه سرشار و گرمای امیدبخش وجودشان، که بهترین پشتیبان من بودند.
% با استفاده از دستور زیر، امضای شما، به طور خودکار، درج می‌شود.
\signature 
\end{acknowledgementpage}}
%%%%%%%%%%%%%%%%%%%%%%%%%%%%%%%%%%%%
% کلمات کلیدی پایان‌نامه را وارد کنید
\keywords{
انتخاب تامین‌کننده، کمی‌سازی ریسک، ارزش در معرض ریسک، ارزش از دست رفته، بهینه‌سازی چندهدفه، برنامه‌ریزی آرمانی، الگوریتم ژنتیک با مرتب‌سازی نامغلوب (نسخه دوم)
}
%چکیده پایان‌نامه را وارد کنید، برای ایجاد پاراگراف جدید از \\ استفاده کنید. اگر خط خالی دشته باشید، خطا خواهید گرفت.
\fa-abstract{در اقتصاد جهانی، شرکت‌ها مرتباً به دنبال تامین‌کنندگان خود در سرتاسر جهان به منظور دستیابی به فرصتی برای کاهش هزینه‌های زنجیره‌تامین هستند. با این حال، تاکید مفرد بر روی هزینه‌های زنجیره‌تامین، می‌تواند زنجیره‌تامین را نسبت به ریسک اختلالات، شکننده و آسیب‌پذیر کند. در این پایان‌نامه، ما از دو نوع مختلف از ریسک استفاده می‌کنیم؛ ارزش در معرض ریسک (\lr{VaR})  و ارزش از دست رفته (\lr{MtT}).
ریسک‌های نوع 
\lr{VaR}
برای مدل کردن حوادث با رخداد کم است که موجب اختلال در عملیات‌های تامین‌کننده و وارد آمدن زیان شدید به خریداران می‌شود (مانند: اعتصابات کارگری، حملات تروریستی، حوادث طبیعی و ...)
. از طرف دیگر، ریسک‌های از نوع ارزش از دست رفته 
(\lr{MtT})
برای مدل کردن حوادث با رخداد بالا برای تامین‌کننده و زیان کمتر به خریدار (مانند: دیرکرد تحویل، از دست دادن الزامات کیفیت و ...)
مورد استفاده قرار می‌گیرند. 
در این پایان‌نامه، ما دو مدل ریاضی چندهدفه تحت دو استراتژی خرید مختلف، ارائه کرده‌ایم. در اولین استراتژی که تک منبعی نامیده می‌شود، فرض می‌شود که خریدار مقدار سفارش برای یک محصول را به یک و فقط یک تامین‌کننده، تخصیص می‌دهد؛ به این معنی که تقسیم کردن سفارش بین تامین‌کنندگان مجاز نمی‌باشد. دومین استراتژی، استراتژی چندمنبعی است که تعمیمی است از مدل تک‌منبعی که خریدار می‌تواند یک سفارش را میان تعدادی از تامین‌کنندگان تقسیم کند. در اولین مدل، ما یک مدل برای تخصیص یک گروهی از تامین‌کنندگان اصلی و چندین تامین‌کننده پشتیبان به یک خریدار، ارائه کرده‌ایم. تامین‌کنندگان اصلی، آن‌هایی هستند که سفارشات را به خریداران تحویل می‌دهند، در حالی که تامین‌کنندگان پشتیبان، تنها زمانی مورد استفاده قرار می‌گیرند که تامین‌کننده اصلی با اختلال مواجه شده است. در دومین مدل، ما از تخفیف بسته‌بندی به عنوان یک شکلی از تخفیف مقداری استفاده کرده‌ایم. بسته‌بندی یک فرمی از تخفیف است که در آن قیمت نهایی یک محصول، به مقادیر سفارش داده شده از محصولات دیگر بستگی دارد. هر دو مدل ارائه شده، دارای سه تابع هدف هستند: حداقل‌سازی هزینه‌ کل، مقدار ارزش در معرض ریسک و ارزش از دست رفته، موارد خریداری شده. این مدل‌ها با استفاده از روش «برنامه‌ریزی آرمانی» و «الگوریتم ژنتیک با مرتب‌سازی نامغلوب نسخه دوم 
\lr{NSGA-2}
» حل شده‌اند و جواب‌هایشان نشان داده شده است.
}



\abstractPage

\newpage\clearpage
\tableofcontents

\newpage
\listoffigures \newpage
\listoftables  \newpage
%\addcontentsline{toc}{chapter}{\listalgorithmname}
%اگر می‌خواهید لیست الگوریتم‌ها را داشته‌ باشید دستور زیر را فعال کنید
%\listofalgorithms \newpage
%اگر می‌خواهید فهرست علایم اختصاری را داشته‌ باشید دستور زیر را فعال کنید
%\chapter*{فهرست علائم اختصاری}
\addcontentsline{toc}{chapter}{فهرست علائم اختصاری}

\persiangloss{\lr{Value At Risk}}{\lr{VaR}}
\persiangloss{ارزش از دست رفته}{\lr{MtT}}
\persiangloss{فرایند تحلیل سلسله مراتبی}{\lr{AHP}}


\pagestyle{fancy}
% اگر شما فصل اول  خود را در فایلی به جز chapter1 همراه با این کلاس نوشته‌اید باید چندخط اول chapter1 را در فایل خود کپی کنید.
% !TeX root=main.tex
% دستور زیر باید در اولین فصل شما باشد. آن را حذف نکنید!
\pagenumbering{arabic}

\chapter{مقدمه} 

در این فصل مطالب مربوط به مقدمه خواهد آمد. 
 
 			% فصل اول: مقدمه
% !TeX root=main.tex
\chapter{پیشینه تحقیق}\label{review}

در اینجا مطالب فصل دوم می‌آید.

























































































		% فصل دوم: آشنایی مقدماتی با لاتک

\chapter{کمی‌سازی ریسک}\label{Risk}


مطالب فصل سوم در اینجا می آید. 





















		% فصل دوم: آشنایی مقدماتی با لاتک
\chapter{مدل‌های چندهدفه}\label{Models}

مطالب مربوط به فصل چهارم اینجا می‌آید.

















































\chapter{نتیجه‌گیری و پیشنهادات}\label{conclution}

مطالب نتیجه‌گیری در این‌جا می‌آید.






























% the references section...

\begin{thebibliography}{9} \label{chapter:references}
\bibitem{}

مهرعلی دهنوی معصومه, آقايی عبداله , ستاك مصطفی, «مديريت ريسک تامين با استفاده از ابزار ارزش در معرض ريسک مبتنی بر تئوری مقدار فرين»،فصلنامه پژوهشنامه بازرگانی، ویژه‌نامه شماره 66، بهار 1392، 161-194.

	\begin{latin}

		\bibitem{Bilsel2011} Bilsel, R.U. and Ravindran, A., 2011. A multiobjective chance constrained programming model for supplier selection under uncertainty. Transportation Research Part B: Methodological, 45(8), pp.1284-1300.
		
		\bibitem{Bilsel} Bilsel, R.U., 2009. Disruption and operational risk quantification and mitigation models for outsourcing operations. The Pennsylvania State University.
		
    		\bibitem{chopra} Chopra, S. and Meindl, P., 2007. Supply chain management. Strategy, planning and operation. In Das summa summarum des management (pp. 265-275). Gabler.
    	
    			\bibitem {Christopher} Christopher, M. and Peck, H., 2004. Building the resilient supply chain. The international journal of logistics management, 15(2), pp.1-14.
    	
    			\bibitem{Coello} Coello, C.A.C., Lamont, G.B. and Van Veldhuizen, D.A., 2007. Evolutionary algorithms for solving multi-objective problems (Vol. 5). New York: Springer.
    			
		\bibitem{DebBook} Deb, K., 2001. Multi-objective optimization using evolutionary algorithms (Vol. 16). John Wiley and Sons.    			
    			
		\bibitem{dey} Dey, P.K., Bhattacharya, A., Ho, W. and Clegg, B., 2015. Strategic supplier performance evaluation: A case-based action research of a UK manufacturing organisation. International Journal of Production Economics, 166, pp.192-214.
	
		\bibitem{Ding} Ding, Z., 2014. Multi-Criteria Multi-Period Supplier Selection and Order Allocation Models.

		\bibitem{Gen} Gen, M. and Cheng, R., 2000. Genetic algorithms and engineering optimization (Vol. 7). John Wiley and Sons.
			
			\bibitem {Ghodsypour} Ghodsypour, S.H. and O’brien, C., 2001. The total cost of logistics in supplier selection, under conditions of multiple sourcing, multiple criteria and capacity constraint. International journal of production economics, 73(1), pp.15-27.
		
		\bibitem{ho} Ho, W., Zheng, T., Yildiz, H. and Talluri, S., 2015. Supply chain risk management: a literature review. International Journal of Production Research, 53(16), pp.5031-5069.
	
			\bibitem{ho2010} Ho, W., Xu, X. and Dey, P.K., 2010. Multi-criteria decision making approaches for supplier evaluation and selection: A literature review. European Journal of operational research, 202(1), pp.16-24.
			
			\bibitem{jolai} Jolai, F., Neyestani, M.S. and Golmakani, H., 2013. Multi-objective model for multi-period, multi-products, supplier order allocation under linear discount. International Journal of Management Science and Engineering Management, 8(1), pp.24-31.
		
		\bibitem{Mendoza} Mendoza, A., 2007. Effective methodologies for supplier selection and order quantity allocation. The Pennsylvania State University.		
		
		
		\bibitem{Ravindran} Ravindran, A.R., Ufuk Bilsel, R., Wadhwa, V. and Yang, T., 2010. Risk adjusted multicriteria supplier selection models with applications. International Journal of Production Research, 48(2), pp.405-424.
		
		\bibitem{Ravindran-2012} Ravindran, A.R. and Warsing Jr, D.P., 2012. Supply chain engineering: Models and applications. CRC Press.
						
		\bibitem{Simchi} Simchi-Levi, D., Simchi-Levi, E. and Kaminsky, P., 1999. Designing and managing the supply chain: Concepts, strategies, and cases. New York: McGraw-Hill.
	
	\bibitem{sum} Sun, Y., 2015. A Multi-period Multi-criteria Supplier Selection and Order Allocation Model Under Demand Uncertainty.	
	
	\bibitem{Tang} Tang, C.S., 2006. Perspectives in supply chain risk management. International journal of production economics, 103(2), pp.451-488.
		
\bibitem{Talluri} Sarkis, J. and Talluri, S., 2002. A model for strategic supplier selection. Journal of supply chain management, 38(4), pp.18-28.		

		\bibitem{Wadhwa} Wadhwa, V., 2008. Multi-objective decision support system for global supplier selection. The Pennsylvania State University.

		\bibitem {Xia}Xia, W. and Wu, Z., 2007. Supplier selection with multiple criteria in volume discount environments. Omega, 35(5), pp.494-504.

	\bibitem{yang} Yang, T., 2006. Multi objective optimization models for managing supply risk in supply chains. The Pennsylvania State University.		
				
	\bibitem{Yoon} Yoon, J., Talluri, S., Yildiz, H. and Ho, W., 2017. Models for supplier selection and risk mitigation: a holistic approach. International Journal of Production Research, pp.1-26.	
	
	
	




		
			

	
	
	
	

		
				
		
		
		
		
		
		
		
		
		
		
		
		
		
		
		
		
		
		
		
		
		
		
		
		
		
		
		
		
		
		
	
		
		% \bibitem{xa-wiki} X/Open XA, \href{http://en.wikipedia.org/wiki/X/Open\_XA}{http://en.wikipedia.org/wiki/X/Open\_XA}.
		
		% \bibitem{odbc-wiki} Open Database Connectivity, \href{http://en.wikipedia.org/wiki/Open\_Database\_Connectivity}{http://en.wikipedia.org/wiki/Open\_Database\_ Connectivity}.
		
	\end{latin}

\end{thebibliography}





% مراجع



\pagestyle{fancy}

\appendix                           %فصلهای پس از این قسمت به عنوان ضمیمه خواهند آمد.
% اگر شما پیوست اول  خود را در فایلی به جز appendix1 همراه با این کلاس نوشته‌اید باید چندخط اول appendix1 را در فایل خود کپی کنید.
%% !TeX root=main.tex
% دستورات زیر باید در اولین فایل پیوست باشند. آنها را حذف نکنید!
\addtocontents{toc}{
    \protect\renewcommand\protect\cftchappresnum{\appendixname~}%
    \protect\setlength{\cftchapnumwidth}{\mylenapp}}%
    
\chapter{مدیریت مراجع در لاتک}\label{App:RefMan}
\thispagestyle{empty}

در بخش \ref{Sec:Ref} اشاره شد که با دستور 
 \lr{\textbackslash bibitem}
  می‌توان یک مرجع را تعریف نمود و با فرمان
 \lr{\textbackslash cite}
  به آن ارجاع داد. این روش برای تعداد مراجع زیاد و تغییرات آنها مناسب نیست. در ادامه به صورت مختصر توضیحی در خصوص برنامه \lr{BibTeX} که همراه با توزیع‌های معروف تِک عرضه می‌شود و نحوه استفاده از آن در زی‌پرشین خواهیم داشت.

\section{ مدیریت مراجع با  \texorpdfstring{\lr{Bib\TeX}}{Bib\TeX} }
یکی از روش‌های قدرتمند و انعطاف‌پذیر برای نوشتن مراجع مقالات و مدیریت مراجع در لاتک، استفاده از  \lr{BibTeX} است.
 روش کار با  \lr{BibTeX} به این صورت است که مجموعه‌ی همه‌ی مراجعی را که در \پ استفاده کرده یا خواهیم کرد، 
در پرونده‌ی جداگانه‌ای نوشته و به آن فایل در سند خودمان به صورت مناسب لینک می‌دهیم.
 کنفرانس‌ها یا مجله‌های گوناگون برای نوشتن مراجع، قالب‌ها یا قراردادهای متفاوتی دارند که به آنها استیلهای مراجع گفته می‌شود.
 در این حالت به کمک ‌استیل‌های \lr{BibTeX} خواهید توانست تنها با تغییر یک پارامتر در پرونده‌ی ورودی خود، مراجع را مطابق قالب موردنظر تنظیم کنید. 
 بیشتر مجلات و کنفرانس‌های معتبر یک پرونده‌ی سبک (\lr{BibTeX Style}) با پسوند \lr{bst} در وب‌گاه خود می‌گذارند که برای همین منظور طراحی شده است.

به جز نوشتن مقالات این سبک‌ها کمک بسیار خوبی برای تهیه‌ی مستندات علمی همچون پایان‌نامه‌هاست که فرد می‌تواند هر قسمت از کارش را که نوشت مراجع مربوطه را به بانک مراجع خود اضافه نماید. با داشتن چنین بانکی از مراجع، وی خواهد توانست به راحتی یک یا چند ارجاع به مراجع و یا یک یا چند بخش را حذف یا اضافه ‌نماید؛ 
مراجع به صورت خودکار مرتب شده و فقط مراجع ارجاع داده شده در قسمت کتاب‌نامه خواهندآمد. قالب مراجع به صورت یکدست مطابق سبک داده شده بوده و نیازی نیست که کاربر درگیر قالب‌دهی به مراجع باشد. 
در این جا مجموعه‌ سبک‌های بسته \lr{Persian-bib} که برای  زی‌پرشین آماده شده‌اند به صورت مختصر معرفی شده و روش کار با آن‌ها گفته می‌شود. برای اطلاع بیشتر به راهنمای بسته‌ی \lr{Persian-bib} مراجعه فرمایید.
\subsection{سبک‌های فعلی قابل استفاده در زی‌پرشین}
در حال حاضر فایلهای سبک زیر برای استفاده در زی‌پرشین آماده شده‌اند:

\singlespacing
\begin{description}
\item [unsrt-fa.bst] این سبک متناظر با \lr{unsrt.bst} می‌باشد. مراجع به ترتیب ارجاع در متن ظاهر می‌شوند.
\item [plain-fa.bst] این سبک متناظر با \lr{plain.bst} می‌باشد. مراجع بر اساس نام‌خانوادگی نویسندگان، به ترتیب صعودی مرتب می‌شوند.
 همچنین ابتدا مراجع فارسی و سپس مراجع انگلیسی خواهند آمد.
\item [acm-fa.bst] این سبک متناظر با \lr{acm.bst} می‌باشد. شبیه \lr{plain-fa.bst} است.  قالب مراجع کمی متفاوت است. اسامی نویسندگان انگلیسی با حروف بزرگ انگلیسی نمایش داده می‌شوند. (مراجع مرتب می‌شوند)
\item [ieeetr-fa.bst] این سبک متناظر با \lr{ieeetr.bst} می‌باشد. (مراجع مرتب نمی‌شوند)
\item [plainnat-fa.bst] این سبک متناظر با \lr{plainnat.bst} می‌باشد. نیاز به بستهٔ \lr{natbib} دارد. (مراجع مرتب می‌شوند)
\item [chicago-fa.bst] این سبک متناظر با \lr{chicago.bst} می‌باشد. نیاز به بستهٔ \lr{natbib} دارد. (مراجع مرتب می‌شوند)
\item [asa-fa.bst] این سبک متناظر با \lr{asa.bst} می‌باشد. نیاز به بستهٔ \lr{natbib} دارد. (مراجع مرتب می‌شوند)
\end{description}
\doublespacing

با استفاده از استیلهای فوق می‌توانید به انواع مختلفی از مراجع فارسی و لاتین ارجاع دهید. به عنوان نمونه مرجع 
\cite{Omidali82phdThesis}
 یک نمونه پروژه دکترا (به فارسی) و مرجع 
\cite{Vahedi87} یک نمونه مقاله مجله فارسی است.
مرجع 
\cite{Amintoosi87afzayesh}  یک نمونه  مقاله کنفرانس فارسی و
مرجع 
\cite{Pedram80osool} یک نمونه کتاب فارسی با ذکر مترجمان و ویراستاران فارسی است. مرجع 
\cite{Khalighi07MscThesis} یک نمونه پروژه کارشناسی ارشد انگلیسی و
\cite{Khalighi87xepersian} هم یک نمونه متفرقه  می‌باشند.

مراجع 
\cite{Gonzalez02book,Baker02limits} 
نمونه کتاب و مقاله انگلیسی هستند.
استیل مورد استفاده در این \پ \lr{acm-fa} است که خروجی آنرا در بخش مراجع می‌توانید مشاهده کنید.
نمونه  خروجی سبک \lr{asa-fa} در شکل \ref{fig:asafa} آمده است.

\begin{figure}[t]
\centering
\includegraphics[width=.8\textwidth]{asa-fa-crop.pdf}
\caption{نمونه خروجی با سبک \lr{asa-fa}}
\label{fig:asafa}
\end{figure} 

\subsection{ نحوه استفاده از سبک‌های فارسی}


برای استفاده از بیب‌تک باید مراجع خود را در یک فایل با پسوند \lr{bib} ذخیره نمایید. یک فایل \lr{bib} در واقع یک پایگاه داده از مراجع\LTRfootnote{Bibliography Database}  شماست که هر مرجع در آن به عنوان یک رکورد از این پایگاه داده
با قالبی خاص ذخیره می‌شود. به هر رکورد یک مدخل\LTRfootnote{Entry} گفته می‌شود. یک نمونه مدخل برای معرفی کتاب \lr{Digital Image Processing} در ادامه آمده است:

\singlespacing
\begin{LTR}
\begin{verbatim}
@BOOK{Gonzalez02image,
  AUTHOR =      {Rafael Gonzalez and Richard Woods},
  TITLE =       {Digital Image Processing},
  PUBLISHER =   {Prentice-Hall, Inc.},
  YEAR =        {2006},
  EDITION =     {3rd},
  ADDRESS =     {Upper Saddle River, NJ, USA}
}
\end{verbatim}
\end{LTR}
\doublespacing

در مثال فوق، \lr{@BOOK} مشخصه‌ی شروع یک مدخل مربوط به یک کتاب و \lr{Gonzalez02book} برچسبی است که به این مرجع منتسب شده است.
 این برچسب بایستی یکتا باشد. برای آنکه فرد به راحتی بتواند برچسب مراجع خود را به خاطر بسپارد و حتی‌الامکان برچسب‌ها متفاوت با هم باشند معمولاً از قوانین خاصی به این منظور استفاده می‌شود. یک قانون می‌تواند فامیل نویسنده‌ی اول+دورقم سال نشر+اولین کلمه‌ی عنوان اثر باشد. به \lr{AUTHOR} و $\dots$ و \lr{ADDRESS} فیلدهای این مدخل گفته می‌شود؛ که هر یک با مقادیر مربوط به مرجع مقدار گرفته‌اند. ترتیب فیلدها مهم نیست. 

انواع متنوعی از مدخل‌ها برای اقسام مختلف مراجع همچون کتاب، مقاله‌ی کنفرانس و مقاله‌ی ژورنال وجود دارد که برخی فیلدهای آنها با هم متفاوت است. 
نام فیلدها بیانگر نوع اطلاعات آن می‌باشد. مثالهای ذکر شده در فایل \lr{MyReferences.bib} کمک خوبی به شما خواهد بود. 
%این فایل یک فایل متنی بوده و با ویرایشگرهای معمول همچون \lr{Notepad++} قابل ویرایش می‌باشد. برنامه‌هایی همچون 
%\lr{TeXMaker}
% امکاناتی برای نوشتن این مدخل‌ها دارند و به صورت خودکار فیلدهای مربوطه را در فایل \lr{bib}  شما قرار می‌دهند.  
با استفاده از سبک‌های فارسی آماده شده، محتویات هر فیلد می‌تواند به فارسی نوشته شود، ترتیب مراجع و نحوه‌ی چینش فیلدهای هر مرجع را سبک مورد استفاده  مشخص خواهد کرد.

نکته: بدون اعمال تنظیمات موردنیاز \lr{Bib\TeX} در \lr{TeXWorks}، مراجع فارسی در استیل‌هایی که مراجع را به صورت مرتب شده چاپ می‌کنند، ترتیب کاملاً درستی نخواهند داشت. برای توضیحات بیشتر \cite{persianbib87userguide} را ببینید یا به سایت پارسی‌لاتک مراجعه فرمایید. تنظیمات موردنیاز در \lr{TeXMaker} اصلاح شده اعمال شده‌اند.

\textbf{برای درج مراجع خود لازم نیست نگران موارد فوق باشید. در فایل 
\lr{MyReferences.bib}
 که همراه با این \پ هست، موارد مختلفی درج شده است و کافیست مراجع خود را جایگزین موارد مندرج در آن نمایید.
}

پس از قرار دادن مراجع خود، یک بار \lr{XeLaTeX} را روی سند خود اجرا نمایید، سپس \lr{bibtex} و پس از آن دوبار \lr{XeLaTeX} را. در \lr{TeXMaker} کلید \lr{F11} و در \lr{TeXWorks} هم گزینه‌ی \lr{BibTeX} از منوی \lr{Typeset}، \lr{BibTeX} را روی سند شما اجرا می‌کنند.

برای بسیاری از مقالات لاتین حتی لازم نیست که مدخل مربوط به آنرا خودتان بنویسید. با جستجوی نام مقاله + کلمه \lr{bibtex}  در اینترنت سایتهای بسیاری همچون \lr{ACM} و \lr{ScienceDirect} را خواهید یافت که مدخل \lr{bibtex} مربوط به مقاله شما را دارند و کافیست آنرا به انتهای فایل \lr{MyReferences} اضافه کنید.

از هر یک از سبکهای \lr{Persian-bib} می‌توانید استفاده کنید، البته اگر از سه استیل آخر استفاده می‌کنید و مایلید که مراجع شما شماره بخورند باید بسته \lr{natbib} را با گزینه \lr{numbers} فراخوانی نمایید.
		% پیوست اول: مدیریت مراجع در لاتک
%% !TeX root=main.tex

\chapter{‌جدول، نمودار و الگوریتم در لاتک}\label{App:Latex:More}
\thispagestyle{empty}

در این بخش نمونه مثالهایی از جدول، نمودار و الگوریتم در لاتک را خواهیم دید.
\section{مدلهای حرکت دوبعدی}
بسیاری از اوقات حرکت بین دو تصویر از یک صحنه با یکی از مدلهای پارامتری ذکر شده در جدول \eqref{tab:MotionModels} قابل مدل نمودن می‌باشد.  
\begin{table}[ht]
\caption{مدلهای تبدیل.}
\label{tab:MotionModels}
\centering
\onehalfspacing
\begin{tabular}{|r|c|l|r|}
\hline نام مدل & درجه آزادی & تبدیل مختصات & توضیح \\ 
\hline انتقالی & ۲ & $\begin{aligned} x'=x+t_x \\ y'=y+t_y \end{aligned}$  &  انتقال دوبعدی\\ 
\hline اقلیدسی & ۳ & $\begin{aligned} x'=xcos\theta - ysin\theta+t_x \\ y'=xsin\theta+ycos\theta+t_y \end{aligned}$  &  انتقالی+دوران \\ 
\hline مشابهت & ۴ & $\begin{aligned} x'=sxcos\theta - sysin\theta+t_x \\ y'=sxsin\theta+sycos\theta+t_y  \end{aligned}$  & اقلیدسی+تغییرمقیاس \\ 
\hline آفین & ۶ & $\begin{aligned} x'=a_{11}x+a_{12}y+t_x \\ y'=a_{21}x+a_{22}y+t_y \end{aligned}$  & مشابهت+اریب‌شدگی \\ 
\hline  پروجکتیو & ۸ & $\begin{aligned} x'&=(m_1x+m_2y+m_3)/D \\ y'&=(m_4x+m_5y+m_6)/D \\ D&=m_7x+m_8y+1 \end{aligned}$  & آفین+\lr{keystone+chirping} \\ 
\hline  شارنوری & $\infty $ & $\begin{aligned} x'=x+v_x(x,y) \\ y'=y+v_y(x,y) \end{aligned}$  &  حرکت آزاد\\ 
\hline 
\end{tabular} 
\end{table}

\section{ماتریس}

شناخته‌شده‌ترین روش تخمین ماتریس هوموگرافی الگوریتم تبدیل خطی مستقیم (\lr{DLT\LTRfootnote{Direct Linear Transform}}) است.  فرض کنید چهار زوج نقطهٔ متناظر در دو تصویر در دست هستند،  $\mathbf{x}_i\leftrightarrow\mathbf{x}'_i$   و تبدیل با رابطهٔ
  $\mathbf{x}'_i = H\mathbf{x}_i$
  نشان داده می‌شود که در آن:
\[\mathbf{x}'_i=(x'_i,y'_i,w'_i)^\top  \]
و
\[ H=\left[
\begin{array}{ccc}
h_1 & h_2 & h_3 \\ 
h_4 & h_5 & h_6 \\ 
h_7 & h_8 & h_9
\end{array} 
\right]\]
رابطه زیر را برای الگوریتم  \eqref{alg:DLT} لازم دارم.
\begin{equation}\label{eq:DLT_Ah}
\left[
\begin{array}{ccc}
0^\top & -w'_i\mathbf{x}_i^\top & y'_i\mathbf{x}_i^\top \\ 
w'_i\mathbf{x}_i & 0^\top & -x'_i\mathbf{x}_i^\top \\ 
- y'_i\mathbf{x}_i^\top & x'_i\mathbf{x}_i^\top & 0^\top
\end{array} 
\right]
\left(
\begin{array}{c}
\mathbf{h}^1 \\ 
\mathbf{h}^2 \\ 
\mathbf{h}^3
\end{array} 
\right)=0
\end{equation}

\section{الگوریتم با دستورات فارسی}
با مفروضات فوق، الگوریتم \lr{DLT} به صورت نشان داده شده در الگوریتم \eqref{alg:DLT}  خواهد بود.
\begin{algorithm}[t]
\onehalfspacing
\caption{الگوریتم \lr{DLT} برای تخمین ماتریس هوموگرافی.} \label{alg:DLT}
\begin{algorithmic}[1]
\REQUIRE $n\geq4$ زوج نقطهٔ متناظر در دو تصویر 
${\mathbf{x}_i\leftrightarrow\mathbf{x}'_i}$،\\
\ENSURE ماتریس هوموگرافی $H$ به نحوی‌که: 
$\mathbf{x}'_i = H \mathbf{x}_i$.
  \STATE برای هر زوج نقطهٔ متناظر
$\mathbf{x}_i\leftrightarrow\mathbf{x}'_i$ 
ماتریس $\mathbf{A}_i$ را با استفاده از رابطهٔ \ref{eq:DLT_Ah} محاسبه کنید.
  \STATE ماتریس‌های ۹ ستونی  $\mathbf{A}_i$ را در قالب یک ماتریس $\mathbf{A}$ ۹ ستونی ترکیب کنید. 
  \STATE تجزیهٔ مقادیر منفرد \lr{(SVD)}  ماتریس $\mathbf{A}$ را بدست آورید. بردار واحد متناظر با کمترین مقدار منفرد جواب $\mathbf{h}$ خواهد بود.
  \STATE  ماتریس هوموگرافی $H$ با تغییر شکل $\mathbf{h}$ حاصل خواهد شد.
\end{algorithmic}
\end{algorithm}

\section{الگوریتم با دستورات لاتین}
الگوریتم \ref{alg:RANSAC} یک الگوریتم با دستورات لاتین است.

\begin{algorithm}[t]
\onehalfspacing
\caption{الگوریتم \lr{RANSAC} برای تخمین ماتریس هوموگرافی.} \label{alg:RANSAC}
\begin{latin}
\begin{algorithmic}[1]
\REQUIRE $n\geq4$ putative correspondences, number of estimations, $N$, distance threshold $T_{dist}$.\\
\ENSURE Set of inliers and Homography matrix $H$.
\FOR{$k = 1$ to $N$}
  \STATE Randomly choose 4 correspondence,
  \STATE Check whether these points are colinear, if so, redo the above step
  \STATE Compute the homography $H_{curr}$ by DLT algorithm from the 4 points pairs,
  \STATE $\ldots$ % الگوریتم کامل نیست
  \ENDFOR
  \STATE Refinement: re-estimate H from all the inliers using the DLT algorithm.
\end{algorithmic}
\end{latin}
\end{algorithm}

\section{نمودار}
لاتک بسته‌هایی با قابلیت‌های زیاد برای رسم انواع مختلف نمودارها دارد. مانند بسته‌های \lr{Tikz} و  \lr{PSTricks}. توضیح اینها فراتر از این پیوست کوچک است. مثالهایی از رسم نمودار را در مجموعه پارسی‌لاتک خواهید یافت. توصیه می‌کنم که حتماً مثالهایی از برخی از آنها را ببینید. راهنمای همه آنها در تک‌لایو هست. نمونه مثالهایی از بسته \lr{Tikz} را می‌توانید در \url{http://www.texample.net/tikz/examples/} ببینید.

\section{تصویر}
نمونه تصاویری در بخش قبل دیدیم. دو تصویر شیر کنار هم را هم در شکل \ref{fig:twolion} مشاهده می‌کنید.
\begin{figure}[t]
\centering 
\subfigure[شیر ۱]{ \label{fig:twolion:one}
\includegraphics[width=.3\textwidth]{lion}}
%\hspace{2mm}
\subfigure[شیر ۲]{ \label{fig:twolion:two}
\includegraphics[width=.3\textwidth]{lion}}
\caption{دو شیر}
\label{fig:twolion} %% label for entire figure
\end{figure}

%\baselineskip=.75cm
\onehalfspacing
%اگر می خواهید واژه نامه فارسی به انگلیسی داشته باشید، دستور زیر را فعال کنید.
%\chapter*{واژه‌نامه فارسی به انگلیسی}\markboth{واژه‌نامه فارسی به انگلیسی}{واژه‌نامه فارسی به انگلیسی}
\addcontentsline{toc}{chapter}{واژه‌نامه فارسی به انگلیسی}
\thispagestyle{empty}

\englishgloss{Probabilistic}{احتمالی}
\englishgloss{Valuation}{ارزیابی}
\englishgloss{Measure}{اندازه }
\englishgloss{Stably}{پایدار}
\englishgloss{Weak Topology}{توپولوژی ضعیف}
\englishgloss{Powerdomain}{دامنه‌توانی}
\englishgloss{Function Space}{فضای تابع}
\englishgloss{Semantic Domain}{دامنه معنایی}
\englishgloss{Program Fragment}{قطعه‌برنامه}
\englishgloss{Dcpo}{مجموعه جزئاً مرتب کامل جهت‌دار}
\englishgloss{Ordered}{مرتب}
%اگر می خواهید واژه نامه انگلیسی به فارسی را داشته باشید، دستور زیر را فعال کنید.
%\chapter*{واژه‌نامه  انگلیسی به  فارسی}\markboth{واژه‌نامه  انگلیسی به  فارسی}{واژه‌نامه  انگلیسی به  فارسی}
\addcontentsline{toc}{chapter}{واژه‌نامه  انگلیسی به  فارسی}
\thispagestyle{empty}

\persiangloss{مجموعه جزئاً مرتب کامل جهت‌دار}{Dcpo}
\persiangloss{فضای تابع}{Function Space}
\persiangloss{اندازه }{Measure}
\persiangloss{مرتب}{Ordered}
\persiangloss{دامنه‌توانی}{Powerdomain}
\persiangloss{احتمالی}{Probabilistic}
\persiangloss{قطعه‌برنامه}{Program Fragment}
\persiangloss{دامنه معنایی}{Semantic Domain}
\persiangloss{پایدار}{Stably}
\persiangloss{ارزیابی}{Valuation}
\persiangloss{توپولوژی ضعیف}{Weak Topology}

\printindex
% !TeX root=main.tex
% در این فایل، عنوان پایان‌نامه، مشخصات خود و چکیده پایان‌نامه را به انگلیسی، وارد کنید.

%%%%%%%%%%%%%%%%%%%%%%%%%%%%%%%%%%%%
\baselineskip=.6cm
\begin{latin}
\latinuniversity{Tafresh University}
\latinfaculty{Ministry of Science, Research and Technology}
\latindepartment{Industrial Engineering Department}
\latinsath{MSc Thesis}
\latintitle{A Risk Adjusted Supplier Selection Model}
\firstlatinsupervisor{Dr. Mohammad Saffari}
%\secondlatinsupervisor{Second Supervisor}
\firstlatinadvisor{Dr. Ali Hossein Mirzaie}
%\secondlatinadvisor{Second Advisor}
\latinname{Miald}
\latinsurname{Rafiee Mashhadi Farahani}
\latinthesisdate{2018}
\latinkeywords{ Supplier Selection; Risk Quantification; Value at Risk; Miss the Target; Multi Objective Optimization; Goal Programming; Nondominated Sorting Genetic Algorithm 2 (NSGA-2) }
\en-abstract{
In the global economy, firms are frequently searching their supplier base around the world to find opportunities for reduce supply chain costs. However, singular emphasis on supply chain cost can make the supply chain brittle and susceptible to the risk of disruptions. In this thesis, we used two different types of risk models, Value at Risk (VaR) and Miss the Target (MtT). VaR type risks are used to model less frequent event which disrupt operations at suppliers and can bring severe impacts to buyers (e.g. labour strike, terrorist attack, natural disaster, etc.). Miss the Target (MtT) type risks, on the other hand, are used to model events that might happen more frequently at suppliers with lesser damage to buyers (e.g. late delivery, missing quality requirement, etc.). In this thesis we propose two multiobjective mathematical models under two different purchasing strategies. The first strategy, called "single sourcing", assumes that the buyer assigns an order for a product to one and only one supplier; that is, order splitting among suppliers is not allowed. The second strategy, called "multiple sourcing" is a generalization of the single sourcing model where the buyer can split  an order among a predetermined number of suppliers. In first model, we have presented a model for assigning a group of primary suppliers and several backup suppliers to a buyer. primary suppliers are those suppliers that ship orders to buyers, whereas backup suppliers are used only when a primary supplier faces disruption. In second model, we used product bundling as a form of quantity discount. Bundling is a form of discount where the final price of a product depends on the quantities of different products ordered. Both models consider three objectives: minimizing total cost, Value at risk value, and Miss the Target value of purchased items. The multiobjective models are solved using "Goal Programming" and "Nondominated Sorting Genetic Algorithm 2 (NSGA-2)" and their solution are illustrated.
}

\latinfirstPage

\end{latin}
\clearpage~
\thispagestyle{empty}


\label{LastPage}

\end{document}